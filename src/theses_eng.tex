
\documentclass[12pt]{article}
\raggedbottom

\usepackage{geometry}
\usepackage[utf8]{inputenc}
\usepackage[hidelinks]{hyperref}
\usepackage{amsmath}
\usepackage{cite}
\usepackage{amsthm}
\usepackage{amssymb}
\usepackage{amsfonts}
\usepackage{mathpartir}
\usepackage{scalerel}
\usepackage{stmaryrd}
\usepackage{authblk}
\usepackage{todonotes}
\presetkeys{todonotes}{inline}{}
\usepackage{bm}
\usepackage{titlesec}
\usepackage{graphicx}

\linespread{1.25}
\geometry{left=3.5cm,right=3.5cm,top=2.5cm,bottom=2.5cm,includehead,includefoot}

\bibliographystyle{alphaurl}

\begin{document}
\clearpage

\begin{titlepage}
    \begin{center}
        \vspace*{1cm}

        {\LARGE \textbf{Type-Theoretic Signatures for Algebraic Theories and Inductive Types}}\\
        \vspace{1em}
        {\large \textsc{Theses of the Ph.D. Dissertation}}

        %% \vspace{0.5cm}
        %% \LARGE
        %% Thesis Subtitle
        \vspace{2em}

        \textit{Author:}\\
        {\large András Kovács}\\
        \vspace{1em}
        \textit{Supervisor:}\\
        {\large Ambrus Kaposi}

        \vfill
        \vspace{4em}

        {\normalsize
        Eötvös Loránd University\\
        Doctoral School of Informatics\\
        Head of School: Erzsébet Csuhaj-Varjú\\
        Foundation and Methodology of Informatics Doctoral Program\\
        Program Director: Zoltán Horváth\\}
        \vspace{1em}
        \includegraphics[width=0.35\textwidth]{elte_cimer_szines}\\
        \vspace{1em}
        {\large September 2021}

    \end{center}
\end{titlepage}
\thispagestyle{empty}

\section{Introduction}

The main goal of the thesis is to develop certain type theories as specification
languages for algebraic theories and inductive types. In each type theory of
signatures presented in the thesis, typing contexts specify algebraic theories
by listing sorts, operations and equations.

The use of dependent type theories as specification languages confers
significant expressiveness and allows us to develop their metatheory using
standard methods from the broader metatheory of type theories.

We present three theories of signatures, in order of increasing expressiveness.
In all three cases, there are further possible variations and design choices.

The current results extend and generalize prior work on signatures for inductive
types, in the context of type theory. A primary motivation for the current
results was to develop more expressive inductive types for future proof
assistants. Thus, our syntaxes and semantics of signatures are close to what
would be required in practical implementation. However, our results can be also
viewed in the broader mathematical context of the study of algebraic theories.

\section{Contributions}

The main contributions are summarized in the following four theses.

\subsection*{Thesis 1}

In Chapter 3 we describe a way to use two-level type theory \cite{twolevel} as a
metalanguage for developing semantics of algebraic signatures. This makes it
possible to work in a concise internal notation of a type theory, and at the
same build semantics internally to arbitrary structured categories. For example,
the signature for natural number objects can be interpreted in any category with
finite products.

\subsection*{Thesis 2}

We present syntax and semantics for finitary quotient inductive-inductive (FQII)
signatures in Chapter 4 of the thesis. These are close in expressive power to
Cartmell's generalized algebraic theories \cite{gat}, but differ in
formalization and what kind of semantics results and constructions are built
around them.
\begin{itemize}
\item FQII signatures can describe most type theories in the wild, thus
      providing a model theory for them through the semantics of signatures.
\item The theory of FQII signatures is specified compactly as a type theory, and
      it is itself amenable to algebraic specification.
\item For each signatures a finitely complete category of algebras is given. This
      category is presented as a cwf (category with families, see \cite{cwfs}) with
      certain type formers, which makes it possible to exactly compute notions of
      induction. We show that induction is equivalent to initiality in each category
      of algebras.
\item We show that initial algebras can be constructed from the syntax of FQII signatures,
      by a term algebra construction. In turn, we show that certain fragments of the syntax
      of FQII signatures can be reduced to basic type formers, thereby reducing some of
      the initial algebras to basic type formers.
\item We show that substitutions of signatures can be viewed as model constructions, being
      functors between categories of algebras in the semantics. Additionally, under the
      assumption that initial FQII-algebras exist, every such functor has a left adjoint.
\end{itemize}

\subsection*{Thesis 3}

In Chapter 5, we modify FQII signatures to obtain infinitary quotient
inductive-inductive signatures. This allows us to describe infinitely branching
trees as initial algebras.
\begin{itemize}
\item Real numbers, surreal numbers, ordinals and the cumulative hierarchy of sets
      \cite{hottbook}
      can be now specified using signatures.
\item Additionally, theories of FQII and infinitary QII signatures can be themselves
      described with infinitary QII signatures. This self-description can be utilized
      to bootstrap the metatheory of theories of signatures, starting from minimal
      assumptions.
\item The semantics of signatures is extended to include \emph{iso-fibrancy} of signature
      types; this means that every construction in the theory of signatures respects
      isomorphisms of described algebras.
\item We adapt constructions of term algebras and left adjoint functors to the current setting.
\item We show that signatures have semantic interpretation internally to the
  theory of signatures itself. This implies, in particular, that for each
  signature, the notion of algebra morphisms can be still specified with a
  signature.
\end{itemize}

\subsection*{Thesis 4}
In Chapter 6, we describe higher inductive-inductive signatures. These differ
from the previous signatures mostly in their intended semantics, whose context
is now homotopy type theory \cite{hottbook}, and which allows specified
equalities to be proof-relevant. The higher-dimensional generalization of types
and equalities makes semantics more complicated, so we only present enough
semantics to specify notions of initiality and induction for each
signature. Additionally, we consider two different notions of algebra morphisms:
one preserves structure strictly (up to definitional equality), while the other
preserves structure up to paths.

\section{Publications}

The above contributions build on and extend the following previous publications,
all coauthored by the thesis' author.
\begin{enumerate}
  \item \emph{A Syntax for Higher Inductive-Inductive Types} \cite{hiit}.
  \item \emph{Signatures and Induction Principles for Higher Inductive-Inductive Types} \cite{hiits}.
  \item \emph{Constructing Quotient Inductive-Inductive Types} \cite{kaposi2019constructing}.
  \item \emph{Large and Infinitary Quotient Inductive-Inductive Types} \cite{iqiit}.
  \item \emph{For Finitary Induction-Induction, Induction is Enough} \cite{ind-ind-reduction}.
\end{enumerate}




\bibliography{references}
\end{document}
