
\documentclass[12pt]{article}
\raggedbottom

\usepackage{geometry}
\usepackage[utf8]{inputenc}
\usepackage[hidelinks]{hyperref}
\usepackage{amsmath}
\usepackage{cite}
\usepackage{amsthm}
\usepackage{amssymb}
\usepackage{amsfonts}
\usepackage{mathpartir}
\usepackage{scalerel}
\usepackage{stmaryrd}
\usepackage{authblk}
\usepackage{todonotes}
\presetkeys{todonotes}{inline}{}
\usepackage{bm}
\usepackage{titlesec}
\usepackage{graphicx}

\linespread{1.25}
\geometry{left=3.5cm,right=3.5cm,top=2.5cm,bottom=2.5cm,includehead,includefoot}

\begin{document}
\clearpage

\begin{titlepage}
    \begin{center}
        \vspace*{1cm}

        {\LARGE \textbf{Type-Theoretic Signatures for Algebraic Theories and Inductive Types}}\\
        \vspace{1em}
        {\large \textsc{Theses of the Ph.D. Dissertation}}

        %% \vspace{0.5cm}
        %% \LARGE
        %% Thesis Subtitle
        \vspace{2em}

        \textit{Author:}\\
        {\large András Kovács}\\
        \vspace{1em}
        \textit{Supervisor:}\\
        {\large Ambrus Kaposi}

        \vfill
        \vspace{4em}

        {\normalsize
        Eötvös Loránd University\\
        Doctoral School of Informatics\\
        Head of School: Erzsébet Csuhaj-Varjú\\
        Foundation and Methodology of Informatics Doctoral Program\\
        Program Director: Zoltán Horváth\\}
        \vspace{1em}
        \includegraphics[width=0.35\textwidth]{elte_cimer_szines}\\
        \vspace{1em}
        {\large September 2021}

    \end{center}
\end{titlepage}
\thispagestyle{empty}

\section{Introduction}

The main goal of the thesis is to develop certain type theories as specification
languages for algebraic theories and inductive types. In each type theory of
signatures presented in the thesis, typing contexts specify algebraic theories
by listing sorts, operations and equations.

The use of dependent type theories as specification languages confers
significant expressiveness and allows us to develop their metatheory using
standard methods from the broader metatheory of type theories.

We present three theories of signatures, in order of increasing expressiveness.
In all three cases, there are further possible variations and design choices.

The current results extend and generalize prior work on signatures for inductive
types, in the context of type theory. A primary motivation for the current
results was to develop more expressive inductive types for future proof
assistants. Thus, our syntaxes and semantics of signatures are close to what
would be required in practical implementation. However, our results can be also
viewed in the broader mathematical context of the study of algebraic theories.


\bibliography{references}
\end{document}
