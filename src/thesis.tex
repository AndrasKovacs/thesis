\documentclass[12pt,a4paper,twoside]{book}

%% build: latexmk -pdf -pvc thesis.tex

%% Packages
%% --------------------------------------------------------------------------------

\usepackage{geometry}
\usepackage[utf8]{inputenc}
\usepackage[hidelinks]{hyperref}
\usepackage{amsmath}
\usepackage{amsthm}
\usepackage{amssymb}
\usepackage{amsfonts}
\usepackage{mathpartir}
\usepackage{scalerel}
\usepackage{stmaryrd}
\usepackage{authblk}
\usepackage{todonotes}
\presetkeys{todonotes}{inline}{}
\usepackage{bm}

%% \bibliographystyle{IEEEtran}
\bibliographystyle{alpha}

\theoremstyle{remark}
\newtheorem{notation}{Notation}

\theoremstyle{definition}
\newtheorem{mydefinition}{Definition}
\newtheorem{myexample}{Example}
\newtheorem{mylemma}{Lemma}

%% Fonts and spacing
%% --------------------------------------------------------------------------------
\linespread{1.25}
%% \geometry{left=3cm,right=2cm,top=2.5cm,bottom=2.5cm,includehead,includefoot}
\geometry{left=4cm,right=2.5cm,top=2.5cm,bottom=2.5cm,includehead,includefoot}


%% Abbrevs
%% --------------------------------------------------------------------------------

\newcommand{\mi}[1]{\mathit{#1}}
\newcommand{\ms}[1]{\mathsf{#1}}

\newcommand{\refl}{\mathsf{refl}}
\newcommand{\id}{\mathsf{id}}
\newcommand{\Con}{\mathsf{Con}}
\newcommand{\Sub}{\mathsf{Sub}}
\newcommand{\Tm}{\mathsf{Tm}}
\newcommand{\Ty}{\mathsf{Ty}}
\newcommand{\U}{\mathsf{U}}
\newcommand{\El}{\mathsf{El}}
\newcommand{\Id}{\mathsf{Id}}
\newcommand{\proj}{\mathsf{proj}}
\renewcommand{\tt}{\mathsf{tt}}
\newcommand{\blank}{\mathord{\hspace{1pt}\text{--}\hspace{1pt}}}
\newcommand{\ra}{\rightarrow}
\newcommand{\Set}{\mathsf{Set}}
\newcommand{\Lift}{\Uparrow}
\newcommand{\ToS}{\mathsf{ToS}}
\newcommand{\ext}{\triangleright}
\newcommand{\emptycon}{\scaleobj{.75}\bullet}
\newcommand{\Pii}{\Pi}
\newcommand{\appi}{\mathsf{app}}
\newcommand{\lami}{\mathsf{lam}}
\newcommand{\Pie}{\Pi^{\mathsf{ext}}}
\newcommand{\appe}{\mathsf{app^{ext}}}
\newcommand{\lame}{\mathsf{lam^{ext}}}
\newcommand{\Piinf}{\Pi^{\mathsf{inf}}}
\newcommand{\appinf}{\mathsf{app^{inf}}}
\newcommand{\laminf}{\mathsf{lam^{inf}}}
\newcommand{\appitt}{\mathop{{\scriptstyle @}}}
\newcommand{\Refl}{\mathsf{Refl}}
\newcommand{\IdU}{\mathsf{IdU}}
\newcommand{\ReflU}{\mathsf{ReflU}}
\newcommand{\Sig}{\mathsf{Sig}}
\newcommand{\ToSSig}{\mathsf{ToSSig}}
\newcommand{\Inductive}{\mathsf{Inductive}}
\newcommand{\Subtype}{\mathsf{Subtype}}
\newcommand{\subtype}{\mathsf{subtype}}
\newcommand{\NatSig}{\mathsf{NatSig}}
\newcommand{\Sg}{\Sigma}
\newcommand{\flCwF}{\mathsf{flCwF}}
\newcommand{\Kfam}{\mathsf{K}}
\newcommand{\p}{\mathsf{p}}
\newcommand{\q}{\mathsf{q}}
\newcommand{\K}{\mathsf{K}}
\newcommand{\lamK}{\mathsf{lam}^{\K}}
\newcommand{\appK}{\mathsf{app}^{\K}}
\newcommand{\A}{\mathsf{A}}
\newcommand{\D}{\mathsf{D}}
\renewcommand{\S}{\mathsf{S}}
\newcommand{\arri}{\Rightarrow}
\newcommand{\arre}{\Rightarrow^{\mathsf{ext}}}
\newcommand{\arrinf}{\Rightarrow^{\mathsf{inf}}}
\newcommand{\syn}{\mathsf{syn}}
\newcommand{\SynSig}{\mathsf{SynSig}}
\newcommand{\bCon}{\boldsymbol{\Con}}
\newcommand{\bTy}{\boldsymbol{\Ty}}
\newcommand{\bSub}{\boldsymbol{\Sub}}
\newcommand{\bTm}{\boldsymbol{\Tm}}
\newcommand{\bGamma}{\boldsymbol{\Gamma}}
\newcommand{\bDelta}{\boldsymbol{\Delta}}
\newcommand{\bsigma}{\boldsymbol{\sigma}}
\newcommand{\bdelta}{\boldsymbol{\delta}}
\newcommand{\bepsilon}{\boldsymbol{\epsilon}}
\newcommand{\bt}{\boldsymbol{t}}
\newcommand{\bu}{\boldsymbol{u}}
\newcommand{\bA}{\boldsymbol{A}}
\newcommand{\ba}{\boldsymbol{a}}
\newcommand{\bb}{\boldsymbol{b}}
\newcommand{\bB}{\boldsymbol{B}}
\newcommand{\bid}{\boldsymbol{\id}}
\newcommand{\bemptycon}{\boldsymbol{\emptycon}}
\newcommand{\bSet}{\boldsymbol{\Set}}
\newcommand{\bU}{\boldsymbol{\U}}
\newcommand{\bEl}{\boldsymbol{\El}}
\newcommand{\bPii}{\boldsymbol{\Pi}}
\newcommand{\bPie}{\boldsymbol{\Pie}}
\newcommand{\bPiinf}{\boldsymbol{\Piinf}}
\newcommand{\bappi}{\boldsymbol{\mathsf{app}}}
\newcommand{\blami}{\boldsymbol{\mathsf{lam}}}
\newcommand{\bId}{\boldsymbol{\Id}}
\newcommand{\bM}{\boldsymbol{\mathsf{M}}}
\newcommand{\bT}{\boldsymbol{\mathsf{T}}}
\newcommand{\bS}{\boldsymbol{\mathsf{S}}}
\newcommand{\bP}{\boldsymbol{\mathsf{P}}}
\newcommand{\bD}{\boldsymbol{\mathsf{D}}}
\newcommand{\bI}{\boldsymbol{\mathsf{I}}}
\newcommand{\ul}[1]{\underline{#1}}
\newcommand{\ulGamma}{\ul{\Gamma}}
\newcommand{\ulDelta}{\ul{\Delta}}
\newcommand{\ulgamma}{\ul{\gamma}}
\newcommand{\ulOmega}{\ul{\Omega}}
\newcommand{\uldelta}{\ul{\delta}}
\newcommand{\ulsigma}{\ul{\sigma}}
\newcommand{\ulnu}{\ul{\nu}}
\newcommand{\ulepsilon}{\ul{\epsilon}}
\newcommand{\ult}{\ul{t}}
\newcommand{\ulu}{\ul{u}}
\newcommand{\ulA}{\ul{A}}
\newcommand{\ula}{\ul{a}}
\newcommand{\ulB}{\ul{B}}
\newcommand{\tos}{\mathsf{tos}}
\newcommand{\coe}{\mathsf{coe}}
\newcommand{\coh}{\mathsf{coh}}
\newcommand{\llb}{\llbracket}
\newcommand{\rrb}{\rrbracket}

\newcommand{\Var}{\ms{Var}}
\newcommand{\var}{\ms{var}}
\newcommand{\app}{\ms{app}}
\newcommand{\vz}{\ms{vz}}
\newcommand{\vs}{\ms{vs}}
\newcommand{\Alg}{\ms{Alg}}
\newcommand{\Mor}{\ms{Mor}}
\newcommand{\DispAlg}{\ms{DispAlg}}
\newcommand{\Section}{\ms{Section}}
\newcommand{\Initial}{\ms{Initial}}

%% --------------------------------------------------------------------------------

\title{Title TBA}
\date{2021 September}
\author{András Kovács}

%% --------------------------------------------------------------------------------

\begin{document}
\maketitle

\frontmatter
\tableofcontents{}


\mainmatter

\chapter{Introduction}

\section{Specification and Semantics for Inductive Types}
\section{Overview of the Thesis and Contributions}
\section{Notation and Conventions}
\label{sec:notation}

\chapter{Simple Inductive Signatures}
\label{chap:simple-inductive-signatures}

In this chapter, we take a look at a very simple notion of inductive
signature. The motivation for doing so is to present the basic ideas of this
thesis in the easiest possible setting. We also include a complete Agda
formalization of the contents of this chapter, in less than 150 lines. Hopefully
this provides intuition for the later chapters, which are greatly generalized
and expanded compared to the current chapter, and which are not feasible (and
probably not that useful) to present in full formal detail.

\todo{potentially in intro}

The mantra throughout this dissertation is the following: inductive types are
specified by typing contexts in certain \emph{theories of signatures}. For each
class of inductive types, there is a corresponding theory of signatures, which
is viewed as a proper type theory and comes equipped with an algebraic model
theory. \emph{Semantics} of signatures is given by interpreting them in certain
models of the theory of signatures. Semantics should at least provide a notion
of induction principle for each signature, but usually we will provide more than
that.

\section{Theory of Signatures}
\label{sec:simple-signatures}

Generally, more expressive theories of signatures can describe a larger classes
of inductive types. As we are aiming at minimalism right now, the current theory
of signatures is as follows:

\begin{mydefinition}The \emph{theory of signatures}, or ToS for short in the current chapter,
is a simple type theory equipped with the following features:
  \begin{itemize}
    \item An empty base type $\iota$.
    \item A \emph{first-order function type} $\iota\!\to\!\blank$; this is a
      function whose domain is fixed to be $\iota$. Moreover, first-order functions only
      have neutral terms: there is application, but no $\lambda$-abstraction.
  \end{itemize}
\end{mydefinition}

We can specify the full syntax using the following Agda-like inductive definitions.
\begin{alignat*}{4}
  & \Ty              &&: \Set           && \Var &&: \Con \to \Ty \to \Set \\
  & \iota            &&: \Ty            && \vz  &&: \Var\,(\Gamma \ext A)\,A \\
  & \iota\!\to\blank &&: \Ty \to \Ty    && \vs  &&: \Var\,\Gamma\,A \to \Var\,(\Gamma \ext B)\,A\\
  & && && &&\\
  & \Con             &&: \Set           && \Tm  &&: \Con \to \Ty \to \Set \\
  & \emptycon        &&: \Con           && \var &&: \Var\,\Gamma\,A \to \Tm\,\Gamma\,A \\
  & \blank\ext\blank &&: \Con \to \Ty \to \Con \hspace{2em} && \app &&: \Tm\,\Gamma\,(\iota\to A) \to \Tm\,\Gamma\,\iota
                                                           \to \Tm\,\Gamma\,A
\end{alignat*}
Here, $\Con$ contexts are lists of types, and $\Var$ specifies well-typed De Bruijn indices, where
$\vz$ represents the zero index, and $\vs$ takes the successor of an index.

\begin{notation} We use capital Greek letters starting from $\Gamma$ to refer to contexts, $A$, $B$, $C$ to
refer to types, and $t$, $u$, $v$ to refer to terms. In examples, we may use a
nameful notation instead of De Bruijn indices. For example, we may write $x :
\Tm\,(\emptycon \ext (x : \iota) \ext (y : \iota))\,\iota$ instead of $\var\,(\vs\,\vz)
: \Tm\,(\emptycon \ext \iota \ext \iota)\,\iota$. Additionally, we may write
$t\,u$ instead of $\app\,t\,u$ for $t$ and $u$ terms.
\end{notation}

\begin{mydefinition} \emph{Parallel substitutions} map variables to terms.
\begin{alignat*}{3}
&\Sub : \Con \to \Con \to \Set\\
&\Sub\,\Gamma\,\Delta \equiv \{A\} \to \Var\,\Delta\,A \to \Tm\,\Gamma\,A
\end{alignat*}
We use $\sigma$ and $\delta$ to refer to substitutions. We also define the action of substitution
on terms, by recursion on terms:
\begin{alignat*}{3}
  &\rlap{$\blank[\blank] : \Tm\,\Delta\,A \to \Sub\,\Gamma\,\Delta \to \Tm\,\Gamma\,A$}\\
  &(\var\, x)   &&[ \sigma ] \equiv \sigma\,x\\
  &(\app\,t\,u) &&[ \sigma ] \equiv \app\,(t[\sigma])\,(u[\sigma])
\end{alignat*}
The \emph{identity substitution} is defined simply as $\id \equiv \var$. It is easy to see that
$t[\id] = t$ for all $t$.
\end{mydefinition}

\begin{myexample} We may write signatures for natural numbers and binary trees respectively as follows.
\begin{alignat*}{3}
  & \ms{NatSig}  &&\equiv \emptycon \ext (\mi{zero} : \iota) \ext (\mi{suc} : \iota \to \iota)\\
  & \ms{TreeSig} &&\equiv \emptycon \ext (\mi{leaf} : \iota) \ext (\mi{node} : \iota \to \iota \to \iota)
\end{alignat*}
\end{myexample}

In short, the current ToS allows inductive types which are
\begin{itemize}
\item \emph{Single-sorted}: this means that we have a single type constructior, corresponding to $\iota$.
\item \emph{Closed}: signatures cannot refer to any externally existing type. For example, we cannot write a signature for ``lists of natural number'' in a direct fashion, since there is no way to refer to the type of natural numbers.
\item \emph{Finitary}: inductive types corresponding to signatures are always
  finitely branching trees. Being closed implies being finitary, since an
  infinitely branching node would require some external type to index subtrees
  with. For example, $\mi{node} : (\mathbb{N} \to \iota) \to \iota$ would
  specify an infinite branching (if such type was allowed in ToS).
\end{itemize}

\emph{Remark.} We omit $\lambda$-expressions from ToS for the sake of
simplicity: this causes terms to be always in normal form (neutral, to be
precise), and thus we can skip talking about conversion rules. Later, starting
from Chapter \ref{chap:fqiit} we include proper $\beta\eta$-rules in signature
theories.

\section{Semantics}
\label{sec:simple-semantics}

For each signature, we need to know what it means for a type theory to support
the corresponding inductive type. For this, we need at least a notion of
\emph{algebras}, which can be viewed as a bundle of all type and
value constructors, and what it means for an algebra to support an
\emph{induction principle}.  Additionally, we may want to know what it means to
support a \emph{recursion principle}, which can be viewed as a non-dependent
variant of induction. In the following, we define these notions by induction on
ToS syntax.

\subsection{Algebras}

First, we calculate types of algebras. This is simply a standard interpretation
into the $\Set$ universe. We define the following operations by induction; the
$\blank^A$ name is overloaded for $\Con$, $\Ty$ and $\Tm$.
\begin{alignat*}{3}
& \hspace{-4em} \rlap{$\blank^A : \Ty \to \Set \to \Set$} \\
& \hspace{-4em} \iota^A\,&&X \equiv X \\
& \hspace{-4em} (\iota\to A)^A\,&&X \equiv X \to A^A\,X\\
& \hspace{-4em} && \\
& \hspace{-4em} \rlap{$\blank^A : \Con \to \Set \to \Set$}\\
& \hspace{-4em} \rlap{$\Gamma^A\,X \equiv \{A : \Ty\} \to \Var\,\Gamma\,A \to A^A\,X$}\\
& \hspace{-4em} && \\
& \hspace{-4em} \rlap{$\blank^A : \Tm\,\Gamma\,A \to \{X : \Set\} \to \Gamma^A\,X \to A^A\,X$}\\
& \hspace{-4em} (\var\,x)^A\,&&\gamma \equiv \gamma\,x\\
& \hspace{-4em} (\app\,t\,u)^A\,&&\gamma \equiv t^A\,\gamma\,(u^A\,\gamma)\\
& \hspace{-4em} && \\
& \hspace{-4em} \rlap{$\blank^A : \Sub\,\Gamma\,\Delta \to \{X : \Set\} \to \Gamma^A\,X \to \Delta^A\,X$}\\
& \hspace{-4em} \rlap{$\sigma^A\,\gamma\,x \equiv (\sigma\,x)^A\,\gamma$}
\end{alignat*}
Here, types and contexts depend on some $X : \Set$, which serves as the
interpretation of $\iota$. We define $\Gamma^A$ as a product: for each variable
in the context, we get a semantic type. This trick, along with the definition of
$\Sub$, makes formalization a bit more compact. Terms and substitutions are
interpreted as natural maps. Substitutions are interpreted by pointwise interpreting
the contained terms.

\begin{notation} We may write $\Gamma^A$ using notation for $\Sigma$-types. For example,
we may write $(\mi{zero} : X) \times (\mi{suc} : X \to X)$ for the result of
computing $\ms{NatSig}^A\,X$.
\end{notation}

\begin{mydefinition} We define \emph{algebras} as follows.
\begin{alignat*}{3}
  & \Alg : \Con \to \Set_1 \\
  & \Alg\,\Gamma \equiv (X : \Set) \times \Gamma^A\,X
\end{alignat*}
\end{mydefinition}

\begin{myexample} $\Alg\,\ms{NatSig}$ is computed to $(X : \Set)\times(\mi{zero} :
X)\times(\mi{suc} : X \to X)$.
\end{myexample}

\subsection{Morphisms}

Now, we compute notions of morphisms of algebras. In this case, morphisms are
functions between underlying sets which preserve all specified structure. The
interpretation for calculating morphisms is a \emph{proof-relevant logical
relation interpretation} \cite{TODO} over the $\blank^A$ interpretation. The key
part is the interpretation of types:
\begin{alignat*}{3}
  & \hspace{-4em}\rlap{$\blank^M : (A : \Ty)\{X_0\,X_1 : \Set\}(X^M : X_0 \to X_1) \to A^A\,X_0 \to A^A\,X_1 \to \Set$}\\
  & \hspace{-4em}\iota^M\,&&X^M\,\alpha_0\,\,\alpha_1 \equiv X^M\,\alpha_0 = \alpha_1 \\
  & \hspace{-4em}(\iota\to A)^M\,&&X^M\,\alpha_0\,\,\alpha_1 \equiv
       (x : X_0) \to A^M\,X^M\,(\alpha_0\,x)\,(\alpha_1\,(X^M\,x))
\end{alignat*}
We again assume an interpretation for the base type $\iota$, as $X_0$, $X_1$ and
$X^M : X_0 \to X_1$. $X^M$ is function between underlying sets of algebras, and
$A^M$ computes what it means that $X^M$ preserves an operation with type $A$. At
the base type, preservation is simply equality. At the first-order function
type, preservation is a quantified statement over $X_0$. We define morphisms for
$\Con$ pointwise:
\begin{alignat*}{3}
  &\Con^M : (\Gamma : \Con)\{X_0\,X_1 : \Set\} \to (X_0 \to X_1) \to \Gamma^A\,X_0 \to \Gamma^A\,X_1 \to \Set\\
  &\Gamma^M\,X^M\,\gamma_0\,\gamma_1 \equiv
    \{A : \Ty\}(x : \Var\,\Gamma\,A) \to A^M\,X^M\,(\gamma_0\,x)\,(\gamma_1\,x)
\end{alignat*}
For terms and substitutions, we get preservation statements, which are sometimes
called \emph{fundamental lemmas} in discussions of logical relations \cite{TODO}.
\begin{alignat*}{3}
  & \hspace{-10em}\rlap{$\blank^M : (t : \Tm\,\Gamma\,A) \to \Gamma^M\,X^M\,\gamma_0\,\gamma_1 \to A^M\,X^M\,(t^A\,\gamma_0)\,(t^A\,\gamma_1)$}\\
  & \hspace{-10em}(\var\,x)^M    &&\gamma^M \equiv \gamma^M\,x \\
  & \hspace{-10em}(\app\,t\,u)^M &&\gamma^M \equiv t^M\,\gamma^M\,(u^A\,\gamma_0)\\
  & \hspace{-10em}&& \\
  & \hspace{-10em}\rlap{$\blank^M : (\sigma : \Sub\,\Gamma\,\Delta) \to \Gamma^M\,X^M\,\gamma_0\,\gamma_1 \to \Delta^M\,X^M\,(\sigma^A\,\gamma_0)\,(\sigma^A\,\gamma_1)$}\\
  & \hspace{-10em} \rlap{$\sigma^M\, \gamma^M\,x = (\sigma\,x)^M\,\gamma^M$}
\end{alignat*}
The definition of $(\app\,t\,u)^M$ is well-typed by the induction hypothesis
$u^M\,\gamma^M : X^M\,(u^A\,\gamma_0) = u^A\,\gamma_1$.

\begin{mydefinition}
We again pack up $\Gamma^M$ with the interpretation of $\iota$, to get notions
of \emph{algebra morphisms}:
\begin{alignat*}{3}
  & \Mor : (\Gamma : \Con) \to \Alg\,\Gamma \to \Alg\,\Gamma \to \Set \\
  & \Mor\,\Gamma\,(X_0,\,\gamma_0)\,(X_1,\,\gamma_1) \equiv
    (X^M : X_0 \to X_1) \times \Gamma^M\,X^M\,\gamma_0\,\gamma_1
\end{alignat*}
\end{mydefinition}
\begin{myexample} We have the following computation:
\begin{alignat*}{3}
  & \hspace{-5em}\rlap{$\Mor\,\NatSig\,(X_0,\,\mi{zero_0},\,\mi{suc_0})\,(X_0,\,\mi{zero_1},\,\mi{suc_1}) \equiv$} \\
           &(X^M : X_0 \to X_1) \\
   \times\,&(X^M\,\mi{zero_0} = \mi{zero_1}) \\
   \times\,&((x : X_0) \to X^M\,(\mi{suc_0}\,x) = \mi{suc_1}\,(X^M\,x))
\end{alignat*}
\end{myexample}

\begin{mydefinition} We state \emph{initiality} as a predicate on algebras:
\begin{alignat*}{3}
  & \Initial : \Alg\,\Gamma \to \Set\\
  & \Initial\,\gamma \equiv
    (\gamma' : \Alg\,\Gamma) \to \ms{isContr}\,(\Mor\,\Gamma\,\gamma\,\gamma')
\end{alignat*}
Here $\ms{isContr}$ refers to unique existence. If we drop $\ms{isContr}$ from
the definition, we get the notion of weak initiality, which corresponds to the
recursion principle for $\Gamma$. Although we call this predicate $\Initial$, in
this chapter we do not yet show that algebras form a category. We provide the
extended semantics in Chapter \ref{chap:fqiit}, and the currently computed
algebras and morphisms remain the same there.
\end{mydefinition}

\paragraph{Morphisms vs.\ logical relations.}
The $\blank^M$ interpretation can be viewed as a special case of logical
relations over the $\blank^A$ model: every morphism is a \emph{functional}
logical relation, where the chosen relation between the underlying sets happens
to be a function. Consider now a more general relational interpretation
for types:
\begin{alignat*}{3}
  & \hspace{-0.5em}\rlap{$\blank^R : (A : \Ty)\{X_0\,X_1 : \Set\}(X^R : X_0 \to X_1 \to \Set) \to A^A\,X_0 \to A^A\,X_1 \to \Set$}\\
  & \hspace{-0.5em}\iota^R\,&&X^R\,\alpha_0\,\,\alpha_1 \equiv X^R\,\alpha_0\,\alpha_1 \\
  & \hspace{-0.5em}(\iota\to A)^R\,&&X^R\,\alpha_0\,\,\alpha_1 \equiv
       (x_0 : X_0)(x_1 : X_1) \to X^R\,x_0\,x_1 \to A^R\,X^R\,(\alpha_0\,x_0)\,(\alpha_1\,x_1)
\end{alignat*}
Here, functions are related if they map related inputs to related outputs. If we
know that $X^M\,\alpha_0\,\alpha_1 \equiv (f\,\alpha_0 = \alpha_1)$ for some $f$
function, we get
\[
  (x_0 : X_0)(x_1 : X_1) \to f\,x_0 = x_1 \to A^R\,X^R\,(\alpha_0\,x_0)\,(\alpha_1\,x_1)
\]
Now, we can simply substitute along the input equality proof in the above type,
to get the previous definition for $(\iota \to A)^M$:
\[
  (x_0 : X_0) \to A^R\,X^R\,(\alpha_0\,x_0)\,(\alpha_1\,(f\,x_0))
\]
This substitution along the equation is called ``singleton contraction'' in the
jargon of homotopy type theory \cite{TODO}. The ability to perform contraction
here is at the heart of the \emph{strict positivity restriction} for inductive
signatures. Strict positivity in our setting corresponds to only having
first-order function types in signatures. If we allowed function domains to be
arbitrary types, in the definition of $(A \to B)^M$ we would only have a
black-box $A^M\,X^M : A^A\,X_0 \to A^A\,X_1 \to \Set$ relation, which is not
known to be given as an equality.

In Chapter \ref{chap:fqiit} we expand on this. As a preliminary summary:
although higher-order functions have relational interpretation, such relations
do not generally compose. What we eventually aim to have is a \emph{category} of
algebras and algebra morphisms, where morphisms properly compose. We need a
\emph{directed} model of the theory of signatures, where every signature becomes
a category of algebras. The way to achieve this, is to prohibit higher-order
functions, thereby avoiding the polarity issues that prevent a directed
interpretation for general function types.

\subsection{Displayed Algebras}

At this point we do not yet have specification for induction principles (or
dependent elimination principles). We use the term \emph{displayed algebra} to
refer to ``dependent'' algebras, where every displayed algebra component lies
over corresponding components in the base algebra. For the purpose of specifying
induction, displayed algebras can be viewed as bundles of induction motives and
methods.

Displayed algebras over some $\gamma : \Alg\,\Gamma$ are equivalent to slices
over $\gamma$ in the category of $\Gamma$-algebras; we show this in Chapter
\ref{chap:fqiit}. A slice $f : \Sub\,\Gamma\,\gamma'\,\gamma$ maps elements of
$\gamma$'s underlying set to elements in the base algebra. Why do we need
displayed algebras, then? The reasons are twofold. First, we need to compute
induction principles exactly, not merely up to isomorphisms, if we are to
eventually implement inductive types in a mechanized setting. Second, displayed
algebras naturally avoid some strictness issues: they support strictly
functorial reindexing, unlike slice objects, whose pullbacks are functorial only
up to isomorphism.

For more illustration of using some displayed algebras in a type-theoretic
setting, see \cite{TODO}. We adapt the term ``displayed algebra'' from ibid.\ as
a generalization of displayed categories and functors.

The displayed algebra interpretation is a \emph{logical predicate}
interpretation, defined as follows.
\begin{alignat*}{3}
  & \rlap{$\blank^D : (A : \Ty)\{X\} \to (X \to \Set) \to A^A\,X \to \Set$}\\
  & \iota^D\,       && X^D\,\alpha \equiv X^D\,\alpha \\
  & (\iota\to A)^D\,&& X^D\,\alpha \equiv (x : X)(x^D : X^D\,x) \to A^D\,X^D\,(\alpha\,x)\\
  & &&\\
  & \rlap{$\blank^D : (\Gamma : \Con)\{X\} \to (X \to \Set) \to \Gamma^A\,X \to \Set$}\\
  & \rlap{$\Gamma^D\,X^D\,\gamma \equiv
       \{A : \Ty\}(x : \Var\,\Gamma\,A) \to A^D\,X^D\,(\gamma\,x)$}\\
  & &&\\
  & \rlap{$\blank^D : (t : \Tm\,\Gamma\,A)
      \to \Gamma^D\,X^D\,\gamma \to A^D\,X^D\,(t^A\,\gamma)$}\\
  & (\var\,x)^D\,&&\gamma^D \equiv \gamma^D\,x\\
  & (\app\,t\,u)^D\,&&\gamma^D \equiv t^D\,\gamma^D\,(u^A\,\gamma)\,(u^D\,\gamma^D)\\
  & &&\\
  & \rlap{$\blank^D : (\sigma : \Sub\,\Gamma\,\Delta)
      \to \Gamma^D\,X^D\,\gamma \to \Delta^D\,X^D\,(\sigma^A\,\gamma)$}\\
  & \rlap{$\sigma^D\,\gamma^D\,x \equiv (\sigma\,x)^D\,\gamma^D$}
\end{alignat*}
Analogously to before, everything depends on a predicate interpretation $X^D : X
\to \Set$ for $\iota$. For types, a predicate holds for a function if the
function preserves predicates. The interpretation of terms is again a
fundamental lemma, and we have pointwise definitions for contexts and
substitutions, as usual.

\begin{myexample} We have the following computation:
\end{myexample}


\subsection{Sections}

\section{Term Algebras}
\subsection{Weak Initiality}
\subsection{Dependent Elimination}

\section{Related and Alternative Approaches}
%% \subsection{F-Algebras}
%% \subsection{Polynomial Functors}

\chapter{Semantics in Two-Level Type Theory}
\label{chap:2ltt}

Introduction: generalizing semantics, distinguishing strict and non-strict equations.
Summary
\begin{itemize}
\item Formal syntax for TT as cwfs, type formers, universes
\item Presheaf models for TT
\item 2LTT
\end{itemize}

\section{Categories with Families}
\label{sec:cwf}

Describe and motivate cwfs for formal syntax. De bruijn indices, examples of
representing stuff. Examples for type formers and universes. Copy from previous papers.

\section{Presheaf Models of Type Theories}
\label{sec:presheaf-models}

\section{Two-Level Type Theory}
\label{sec:2ltt}

\subsection{Models}
\subsection{Properties}

\section{Simple Inductive Signatures}
\subsection{Internal Semantics}
\subsection{Strict and Weak Morphisms}
\subsection{Internal Term Algebras}

\begin{itemize}
\item AMDS
\item finite product semantics, computed examples
\item Inner term algebras.
\item Weak initiality
\item Dependent elimination
\end{itemize}

\chapter{Finitary Quotient Inductive-Inductive Types}
\label{chap:fqiit}

\section{Theory of Signatures}
\label{sec:fqiit-tos}

\subsection{Models}
\subsection{Examples}

\section{Semantics}
\label{sec:fqiit-semantics}
\subsection{Finite Limit Cwfs}
\subsection{Equivalence of Initiality and Induction}
\subsection{Model of the Theory of Signatures}

\section{Term Algebras}
\subsection{Generic Term Algebras}
\subsection{Induction for Term Algebras}
\subsection{Church Encodings}
\subsection{Awodey-Frey-Speight Encodings}

\section{Left Adjoints of Signature Morphisms}

\chapter{Infinitary Quotient Inductive-Inductive Types}
\section{Theory of Signatures}
\section{Term Algebras}

\chapter{Levitation, Bootstrapping and Universe Levels}
\section{Levitation for Closed QIITs}
\section{Levitation for Infinitary QIITs}

\chapter{Higher Inductive-Inductive Types}

\section{Theory of Signatures}
\section{Semantics}

\chapter{Reductions}

\section{Finitary Inductive Types}
\section{Finitary Inductive-Inductive Types}
\section{Closed Quotient Inductive-Inductive Types}

\chapter{Conclusion}


\backmatter
\end{document}
