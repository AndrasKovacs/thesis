

\documentclass[12pt]{article}
\raggedbottom

\usepackage{geometry}
\usepackage[utf8]{inputenc}
\usepackage[hidelinks]{hyperref}
\usepackage{amsmath}
\usepackage{cite}
\usepackage{amsthm}
\usepackage{amssymb}
\usepackage{amsfonts}
\usepackage{mathpartir}
\usepackage{scalerel}
\usepackage{stmaryrd}
\usepackage{authblk}
\usepackage{todonotes}
\presetkeys{todonotes}{inline}{}
\usepackage{bm}
\usepackage{titlesec}
\usepackage{graphicx}

\linespread{1.25}
\geometry{left=3.5cm,right=3.5cm,top=2.5cm,bottom=2.5cm,includehead,includefoot}

\bibliographystyle{alphaurl}

\begin{document}
\clearpage

\begin{titlepage}
    \begin{center}
        \vspace*{1cm}

        {\LARGE \textbf{Algebrai elméletek és induktív típusok specifikációja típuselméleti szignatúrákkal}}\\
        \vspace{1em}
        {\large \textsc{A Ph.D. disszertáció tézisei}}

        %% \vspace{0.5cm}
        %% \LARGE
        %% Thesis Subtitle
        \vspace{2em}

        \textit{Szerző:}\\
        {\large Kovács András}\\
        \vspace{1em}
        \textit{Témavezető:}\\
        {\large Kaposi Ambrus}

        \vfill
        \vspace{4em}

        {\normalsize
        Eötvös Loránd Tudományegyetem\\
        Informatika Doktori Iskola\\
        Doktori iskola vezetője: Csuhaj-Varjú Erzsébet\\
        Doktori program: Az informatika alapjai és módszertana\\
        Programvezető: Horváth Zoltán\\}
        \vspace{1em}
        \includegraphics[width=0.35\textwidth]{elte_cimer_szines}\\
        \vspace{1em}
        {\large 2021 szeptember}

    \end{center}
\end{titlepage}
\thispagestyle{empty}

\section{Bevezető}

A tézis fő célja az, hogy kidolgozza bizonyos típuselméletek használatát
algebrai elméletek és induktív típusok leírásához. Minden ilyen típuselméletben
a típuskörnyezeteket értelmezzük algebrai szignatúraként, ami felsorolja
egy algebrai elmélet szortjait, műveleteit és egyenleteit.

A függő típuselméletek kifejezőereje nagyban elősegíti a tömör és általános
specifikációkat, és lehetővé teszi, hogy a szigatúrák szemantikáját és
metaelméletét olyan eszközökkel viszgáljuk, amelyek korábbról ismertek a
típuselméletben.

Három szignatúra-elméletet mutatunk be. Mindhárom esetben lehetőség van az
elméletek kisebb változtatásaira.

A jelenlegi kutatás kiegészíti és általánosítja az induktív szignatúrák korábbi
irodalmát a típuselmélet kontextusában. A kutatásunk egyik fontos motivációja az
volt, hogy nagy kifejezőerejű induktív típusokat fejlesszünk jövőbeli
tételbizonyító-rendszerekhez. Ebből kifolyólag a szignatúráink szintaxisa és
szemantikája közel van ahhoz, ami praktikus rendszerekben lenne
szükséges. Ugyanakkor az eredményeink felhasználhatók általánosabb matematikai
kontextusban, az algebrai elméletek kutatásában.

\section{Contributions}

A fő eredményeket a következőkben foglaljuk össze.

\subsection*{Thesis 1}

In Chapter 3 we describe a way to use two-level type theory \cite{twolevel} as a
metalanguage for developing semantics of algebraic signatures. This makes it
possible to work in a concise internal notation of a type theory, and at the
same build semantics internally to arbitrary structured categories. For example,
the signature for natural number objects can be interpreted in any category with
finite products.

\subsection*{Thesis 2}

We present syntax and semantics for finitary quotient inductive-inductive (FQII)
signatures in Chapter 4 of the thesis. These are close in expressive power to
Cartmell's generalized algebraic theories \cite{gat}, but differ in
formalization and what kind of semantics results and constructions are built
around them.
\begin{itemize}
\item FQII signatures can describe most type theories in the wild, thus
      providing a model theory for them through the semantics of signatures.
\item The theory of FQII signatures is specified compactly as a type theory, and
      it is itself amenable to algebraic specification.
\item For each signatures a finitely complete category of algebras is given. This
      category is presented as a cwf (category with families, see \cite{cwfs}) with
      certain type formers, which makes it possible to exactly compute notions of
      induction. We show that induction is equivalent to initiality in each category
      of algebras.
\item We show that initial algebras can be constructed from the syntax of FQII signatures,
      by a term algebra construction. In turn, we show that certain fragments of the syntax
      of FQII signatures can be reduced to basic type formers, thereby reducing some of
      the initial algebras to basic type formers.
\item We show that substitutions of signatures can be viewed as model constructions, being
      functors between categories of algebras in the semantics. Additionally, under the
      assumption that initial FQII-algebras exist, every such functor has a left adjoint.
\end{itemize}

\subsection*{Thesis 3}

In Chapter 5, we modify FQII signatures to obtain infinitary quotient
inductive-inductive signatures. This allows us to describe infinitely branching
trees as initial algebras.
\begin{itemize}
\item Real numbers, surreal numbers, ordinals and the cumulative hierarchy of sets
      \cite{hottbook}
      can be now specified using signatures.
\item Additionally, theories of FQII and infinitary QII signatures can be themselves
      described with infinitary QII signatures. This self-description can be utilized
      to bootstrap the metatheory of theories of signatures, starting from minimal
      assumptions.
\item The semantics of signatures is extended to include \emph{iso-fibrancy} of signature
      types; this means that every construction in the theory of signatures respects
      isomorphisms of described algebras.
\item We adapt constructions of term algebras and left adjoint functors to the current setting.
\item We show that signatures have semantic interpretation internally to the
  theory of signatures itself. This implies, in particular, that for each
  signature, the notion of algebra morphisms can be still specified with a
  signature.
\end{itemize}

\subsection*{Thesis 4}
In Chapter 6, we describe higher inductive-inductive signatures. These differ
from the previous signatures mostly in their intended semantics, whose context
is now homotopy type theory \cite{hottbook}, and which allows specified
equalities to be proof-relevant. The higher-dimensional generalization of types
and equalities makes semantics more complicated, so we only present enough
semantics to specify notions of initiality and induction for each
signature. Additionally, we consider two different notions of algebra morphisms:
one preserves structure strictly (up to definitional equality), while the other
preserves structure up to paths.

\section{Publications}

The above contributions build on and extend the following previous publications,
all coauthored by the thesis' author.
\begin{enumerate}
  \item \emph{A Syntax for Higher Inductive-Inductive Types} \cite{hiit}.
  \item \emph{Signatures and Induction Principles for Higher Inductive-Inductive Types} \cite{hiits}.
  \item \emph{Constructing Quotient Inductive-Inductive Types} \cite{kaposi2019constructing}.
  \item \emph{Large and Infinitary Quotient Inductive-Inductive Types} \cite{iqiit}.
  \item \emph{For Finitary Induction-Induction, Induction is Enough} \cite{ind-ind-reduction}.
\end{enumerate}




\bibliography{references}
\end{document}
