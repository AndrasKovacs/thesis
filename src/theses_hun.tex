

\documentclass[12pt]{article}

\usepackage{geometry}
\usepackage[utf8]{inputenc}
\usepackage[hidelinks]{hyperref}
\usepackage{amsmath}
\usepackage{cite}
\usepackage{amsthm}
\usepackage{amssymb}
\usepackage{amsfonts}
\usepackage{mathpartir}
\usepackage{scalerel}
\usepackage{stmaryrd}
\usepackage{authblk}
\usepackage{todonotes}
\presetkeys{todonotes}{inline}{}
\usepackage{bm}
\usepackage{titlesec}
\usepackage{graphicx}

\linespread{1.25}
\geometry{left=3.5cm,right=3.5cm,top=2.5cm,bottom=2.5cm,includehead,includefoot}

\bibliographystyle{alphaurl}

\begin{document}
\clearpage

\begin{titlepage}
    \begin{center}
        \vspace*{1cm}

        {\LARGE \textbf{Algebrai elméletek és induktív típusok specifikációja típuselméleti szignatúrákkal}}\\
        \vspace{1em}
        {\large \textsc{A Ph.D. disszertáció tézisei}}

        %% \vspace{0.5cm}
        %% \LARGE
        %% Thesis Subtitle
        \vspace{2em}

        \textit{Szerző:}\\
        {\large Kovács András}\\
        \vspace{1em}
        \textit{Témavezető:}\\
        {\large Kaposi Ambrus}

        \vfill
        \vspace{4em}

        {\normalsize
        Eötvös Loránd Tudományegyetem\\
        Informatika Doktori Iskola\\
        Doktori iskola vezetője: Csuhaj-Varjú Erzsébet\\
        Doktori program: Az informatika alapjai és módszertana\\
        Programvezető: Horváth Zoltán\\}
        \vspace{1em}
        \includegraphics[width=0.35\textwidth]{elte_cimer_szines}\\
        \vspace{1em}
        {\large 2022 március}

    \end{center}
\end{titlepage}
\thispagestyle{empty}

\section{Bevezető}

A tézis fő célja az, hogy kidolgozza bizonyos típuselméletek használatát
algebrai elméletek és induktív típusok leírásához. Minden ilyen típuselméletben
a típuskörnyezeteket értelmezzük algebrai szignatúraként, ami felsorolja
egy algebrai elmélet szortjait, műveleteit és egyenleteit.

A függő típuselméletek kifejezőereje nagyban elősegíti a tömör és általános
specifikációkat, és lehetővé teszi, hogy a szigatúrák szemantikáját és
metaelméletét olyan eszközökkel viszgáljuk, amelyek korábbról ismertek a
típuselméletben.

Három szignatúra-elméletet mutatunk be. Mindhárom esetben lehetőség van az
elméletek kisebb változtatásaira.

A jelenlegi kutatás kiegészíti és általánosítja az induktív szignatúrák korábbi
irodalmát a típuselmélet kontextusában. A kutatásunk egyik fontos motivációja az
volt, hogy nagy kifejezőerejű induktív típusokat fejlesszünk jövőbeli
tételbizonyító-rendszerekhez. Ebből kifolyólag a szignatúráink szintaxisa és
szemantikája közel van ahhoz, ami praktikus rendszerekben lenne
szükséges. Ugyanakkor az eredményeink felhasználhatók általánosabb matematikai
kontextusban, az algebrai elméletek kutatásában.

\section{Eredmények}

A fő eredményeket a következőkben foglaljuk össze.

\subsection*{1. Tézis}

A harmadik fejezetben kifejtjük, hogy a kétszintű típuselmélet \cite{twolevel}
hogyan használható metanyelvként az algebrai szignatúrák szemantikájához. Ez
lehetővé teszi, hogy a szemantikát általánosan adjuk meg, internálisan tetszőleges
strukturált kategóriákban, és ugyanakkor tömör típuselméleti nyelvben
dolgozzunk. Például a természetes szám objektumok szignatúrája értelmezhető
tetszőleges olyan kategóriában, ami rendelkezik véges szorzatokkal.

\subsection*{2. Tézis}

A tézis negyedik fejezetében bemutatjuk a véges aritású kvóciens
induktív-induktív (FQII, ``finitary quotient inductive-inductive'') szignatúrák
szintaxisát és szemantikáját. Ezek a szignatúrák közel vannak kifejezőerő
tekintetében Cartmell általánosított algebrai elméleteihez \cite{gat}, viszont
jelentős különbségek vannak a formalizációban és a szemantikai konstrukciókban
és eredményekben.
\begin{itemize}
\item Az FQII szignatúrák képesek leírni szinte az összes típuselméletet,
      és így modell-elméletet kapunk hozzájuk, a szignatúrák szemantikáján
      keresztül.
\item Az FQII szignatúrák elmélete tömören specifikált mint típuselmélet, és
      maga is algebrai elmélet.
\item
      Minden szignatúrához egy véges limitekkel rendelkező kategóriát rendelünk,
      amelynek az objektumai algebrák. Ezt a kategóriát egy családos
      kategóriaként \cite{cwfs} prezentáljuk, ami lehetővé teszi, hogy precízen
      kiszámoljuk az indukció fogalmát. Megmutatjuk, hogy az indukció ekvivalens
      az inicialitással minden szignatúra esetén.
\item
     Megmutatjuk, hogy az iniciális algebrákat meg lehet konstruálni az FQII
     szignatúrák szintaxisából, egy term modell konstrukcióval. Továbbá
     megmutatjuk, hogy az FQII szignatúrák szintaxisának bizonyos töredékei
     megkonstruálhatók egyszerűbb típusokból.
\item
     Megmutatjuk, hogy a szignatúrák közötti párhuzamos helyettesítésekre modell
     konstrukciókként lehet gondolni, mivel a szemantikában funktorok lesznek
     algebrák kategóriái között.  Továbbá, feltéve hogy minden iniciális
     FQII-algebra létezik, minden ilyen funktor jobb adjungált.
\end{itemize}

\subsection*{3. Tézis}

Az tézis ötödik fejezetében módosítjuk az FQII szignatúrákat úgy, hogy
végtelenül elágazó fa struktúrákat is le tudjunk írni az iniciális
algebrákban. Így kapjuk a végtelen aritású QII szignatúrák elméletét (IQII
röviden, ``infinitary quotient inductive-inductive'').
\begin{itemize}
\item Valós számok, szürreális számok, ordinálisok és a kumulatív halmaz-hierarchiák
  leírhatók IQII szignatúrák segítségével \cite{hottbook}, ami a véges aritású
  esetben nem volt lehetséges.
\item
  Az FQII és IQII szignatúrák elméletei egyaránt leírhatók IQII szignatúrával.
  A IQII szignatúrák tehát a saját elméletüket is specifikálják, és ezt arra használjuk,
  hogy minimalizáljuk a szükséges feltételezéseket az IQII szignatúrák metaelméletének
  kidolgozásánál.
\item
  A szignatúrák szemantikáját kibővítjük a \emph{izo-fibrálás} tulajdonsággal a
  szignatúra-típusokban. Ez azt jelenti, hogy a szignatúrák elméletében minden
  konstrukció megőrzi a leírt algebrák izomorfizmusait.
\item
  Adaptáljuk a term algebra konstrukciót és a bal adjungált funktorok konstrukcióját
  a végtelen aritású esetre.
\item
  Megmutatjuk, hogy a szignatúrákat lehet szemantikusan értelmezni magában a
  szignatúrák elméletének szintaxisában is. Egy példát hozva, ez azt
  eredményezi, hogy minden szignatúrához megkapjuk az algebra-homomorfizmusok
  specifikációját is szignatúraként.
\end{itemize}

\subsection*{4. Tézis}
A hatodik fejezetben leírjuk a magasabb induktív-induktív szignatúrákat. Ezek
elsősorban a szemantikában különbözek a korábbi szignatúráktól: a metanyelv most
a homotópia típuselmélet \cite{hottbook}. Míg korábban kizárólag egy-dimenziós
egyenleteket adhattunk szignatúrákhoz, most tetszőleges magasabb-dimenziós
utakat tudunk specifikálni, az iniciális algebrák pedig szabadon generált
omega-groupoidokat adnak meg. A magasabb-dimenziós általánosítás jelentősen
bonyolítja a szemantikát, ezért éppen csak annyi szemantikát adunk meg, amiből
az iniciálitás és indukció fogalmai következnek (minden szignatúrához). Továbbá,
az algebra-morfizmusok két változatát kezeljük: az első szigorúan őrzi meg a
struktúrákat, azaz definícionális egyenletekkel, míg a másik gyengén, azaz
a belső intenzionális egyenlőségekkel.

\section{Publikációk}

A fenti eredmények a következő publikációk tartalmára építenek, amelyek
társszerzője a jelenlegi tézis szerzője.
\begin{enumerate}
  \item \emph{A Syntax for Higher Inductive-Inductive Types} \cite{hiit}.
  \item \emph{Signatures and Induction Principles for Higher Inductive-Inductive Types} \cite{hiits}.
  \item \emph{Constructing Quotient Inductive-Inductive Types} \cite{kaposi2019constructing}.
  \item \emph{Large and Infinitary Quotient Inductive-Inductive Types} \cite{iqiit}.
  \item \emph{For Finitary Induction-Induction, Induction is Enough} \cite{ind-ind-reduction}.
\end{enumerate}


\bibliography{references}
\end{document}
